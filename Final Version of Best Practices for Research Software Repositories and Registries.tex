\documentclass[]{article}
\usepackage{lmodern}
\usepackage{amssymb,amsmath}
\usepackage{ifxetex,ifluatex}
\usepackage{fixltx2e} % provides \textsubscript
\ifnum 0\ifxetex 1\fi\ifluatex 1\fi=0 % if pdftex
  \usepackage[T1]{fontenc}
  \usepackage[utf8]{inputenc}
\else % if luatex or xelatex
  \ifxetex
    \usepackage{mathspec}
  \else
    \usepackage{fontspec}
  \fi
  \defaultfontfeatures{Ligatures=TeX,Scale=MatchLowercase}
\fi
% use upquote if available, for straight quotes in verbatim environments
\IfFileExists{upquote.sty}{\usepackage{upquote}}{}
% use microtype if available
\IfFileExists{microtype.sty}{%
\usepackage{microtype}
\UseMicrotypeSet[protrusion]{basicmath} % disable protrusion for tt fonts
}{}
\usepackage[unicode=true]{hyperref}
\hypersetup{
            pdftitle={Best Practices for Research Software Registries and Repositories: A Concise Guide},
            pdfborder={0 0 0},
            breaklinks=true}
\urlstyle{same}  % don't use monospace font for urls
\IfFileExists{parskip.sty}{%
\usepackage{parskip}
}{% else
\setlength{\parindent}{0pt}
\setlength{\parskip}{6pt plus 2pt minus 1pt}
}
\setlength{\emergencystretch}{3em}  % prevent overfull lines
\providecommand{\tightlist}{%
  \setlength{\itemsep}{0pt}\setlength{\parskip}{0pt}}
\setcounter{secnumdepth}{0}
% Redefines (sub)paragraphs to behave more like sections
\ifx\paragraph\undefined\else
\let\oldparagraph\paragraph
\renewcommand{\paragraph}[1]{\oldparagraph{#1}\mbox{}}
\fi
\ifx\subparagraph\undefined\else
\let\oldsubparagraph\subparagraph
\renewcommand{\subparagraph}[1]{\oldsubparagraph{#1}\mbox{}}
\fi

\title{\protect\hypertarget{_n68im9e6pq4}{}{}Best Practices for Research
Software Registries and Repositories: A Concise Guide}
\date{}

\begin{document}
\maketitle

\textbf{TABLE OF CONTENTS}

\section{Introduction}\label{introduction}

Scientific software registries and repositories serve various roles in
their respective disciplines. \emph{Registries} are typically indexes or
catalogs of software stored elsewhere, while \emph{repositories} are
both indexes \emph{and} places where software is stored. Both types of
resource improve software discoverability and research transparency,
provide information for software citations, and foster preservation of
computational methods that might otherwise be lost over time, thereby
supporting research reproducibility and replicability. Many provide or
are integrated with other services, including indexing and archival
services, which can be leveraged by librarians, digital archivists,
journal editors and publishers, and researchers alike.

Having specific policies in place for software registries and
repositories ensures that users and administrators have reference
documents to help define a shared understanding of the scope, practices,
and rules that govern these collections. These practices can prove
useful in a variety of situations, including, but not limited to,
presenting the contents in the resource to stakeholders and community
members, reassuring potential contributors by clarifying sensitive
issues such as attribution, and defining how content in a registry or
repository can be (re)used by others.

The best practices presented here were proposed and developed by a Task
Force of the
\href{https://github.com/force11/force11-sciwg}{\emph{FORCE11 Software
Citation Implementation Working Group}} that was composed of managers
and editors of scientific software registries and repositories during a
series of monthly conference calls in 2019. The Task Force's results
were further developed during the
\href{https://asclnet.github.io/SWRegistryWorkshop/}{\emph{Scientific
Software Registry Collaboration Workshop}}, a two-day workshop held
November 13-14, 2019 at the University of Maryland before being refined
further in virtual meetings and through online collaborative writing in
2020.

Each guideline is presented below with an explanation as to why we
recommend the practice, what the practice describes or contains, and
specific considerations to take into account. Examples of each practice
are provided at the end of the document. Throughout this guide, we refer
to registries and repositories collectively as ``resources'' and
``collections.''

Although our recommendations are partitioned into nine separate policies
or statements, there is inevitable overlap between some of them. In
practice, the statements and policies are often combined into a smaller
number of documents, as most of our real-world examples demonstrate.

This guide is not an exhaustive list of best practices, but we feel that
adopting those suggested here will help better document, guide, and
preserve scientific software registries and repositories, putting them
in a stronger position to serve their disciplines, users, and
communities.

\newpage
\section{Best Practice: Provide a public scope
statement}\label{best-practice-provide-a-public-scope-statement}

\textbf{Why we recommend this}: A scope statement clarifies the type of
software contained in the repository or indexed in the registry. This
manages the expectations of the potential depositor of metadata and/or
software, as well as the resource seeker. It informs both of what the
collection does and does not contain.

\textbf{This should describe}:

\begin{itemize}
\item
  \begin{quote}
  What is accepted, and acceptable
  \end{quote}
\item
  \begin{quote}
  What is not accepted
  \end{quote}
\item
  \begin{quote}
  Exceptions to either/both of the above if necessary
  \end{quote}
\end{itemize}

\textbf{What you might consider when writing a scope statement: }

\begin{itemize}
\item
  \begin{quote}
  The types of software listed in the registry or stored in the
  repository, such as source code or compiled executables
  \end{quote}
\item
  \begin{quote}
  Defining the community being served
  \end{quote}
\item
  \begin{quote}
  Criteria that must be satisfied by accepted software, such as whether
  certain software quality metrics must be fulfilled or whether the
  software must be used in published research
  \end{quote}
\item
  \begin{quote}
  Whether the code has to be in the public domain and/or have a license
  from a predefined set
  \end{quote}
\item
  \begin{quote}
  Whether software registered in another registry or repository will be
  accepted
  \end{quote}
\end{itemize}

\section{}\label{section}

\section{}\label{section-1}

\section{}\label{section-2}

\section{\texorpdfstring{\\
}{ }}\label{section-3}

\newpage
\section{Best Practice: Provide guidance for users
}\label{best-practice-provide-guidance-for-users}

\textbf{Why we recommend this}: Different users of the registry or
repository will benefit from guidance on how to access the information
they are interested in. For example, it is useful to describe how to
search the collection, answer frequently asked questions (FAQs), provide
tips and tricks, and to let users know who to contact for
assistance.\\[2\baselineskip]A separate section in these guidelines on
the \emph{Conditions of use policy} covers terms of use of the
collection, including data and API, and how best to cite records in the
resource and the resource itself. Guidance for users who wish to
contribute software is covered in the next section, \emph{Provide
guidance to software contributors}.

\textbf{This should describe}:

\begin{itemize}
\item
  \begin{quote}
  How to perform common user tasks
  \end{quote}
\item
  \begin{quote}
  Answers to questions that are often asked or can be anticipated
  \end{quote}
\item
  \begin{quote}
  Whom to contact for questions or help
  \end{quote}
\end{itemize}

\textbf{What you might consider when writing guidance for users:}

\begin{itemize}
\item
  \begin{quote}
  Identifying the types of users your resource has or could potentially
  have, and corresponding use cases
  \end{quote}
\item
  \begin{quote}
  Offering multiple forms of guidance, such as in-field prompts, linked
  explanations, and completed examples
  \end{quote}
\item
  \begin{quote}
  If there is an API, including a description specifying the interface
  or a pointer to the official documentation for the interface
  \end{quote}
\item
  \begin{quote}
  If content negotiation is enabled, stating what formats, such as
  JSON-LD or XML, are supported
  \end{quote}
\end{itemize}

\section{\texorpdfstring{\\
}{ }}\label{section-4}

\newpage
\section{Best Practice: Provide guidance to software
contributors}\label{best-practice-provide-guidance-to-software-contributors}

\textbf{Why we recommend this:} People interested in contributing
software entries to the registry or repository need to know what the
process entails. The scope statement will already have explained
\emph{what} is accepted and what is not; the contributor policy
addresses \emph{who} can add or change software entries, and the
processes involved.

\textbf{This should describe}:

\begin{itemize}
\item
  \begin{quote}
  Who can or cannot submit entries and/or metadata
  \end{quote}
\item
  \begin{quote}
  Required and optional metadata expected from software contributors
  \end{quote}
\item
  \begin{quote}
  Review process, if any
  \end{quote}
\item
  \begin{quote}
  Curation process, if any
  \end{quote}
\item
  \begin{quote}
  Procedures for updates (e.g., who can do it, when it is done, how is
  it done)
  \end{quote}
\end{itemize}

\textbf{What you might consider when writing a contributor policy}:

\begin{itemize}
\item
  \begin{quote}
  Defining who can submit and/or update entries
  \end{quote}
\item
  \begin{quote}
  Whether the author(s) of the software will be contacted if the
  contributor is not also an author, and whether contact is a condition
  or side-effect of the submission
  \end{quote}
\item
  \begin{quote}
  Stating how persistent identifiers, if used, are assigned
  \end{quote}
\item
  \begin{quote}
  Including a statement that depositors must comply with all applicable
  laws and not be intentionally malicious
  \end{quote}
\end{itemize}

\section{\texorpdfstring{\\
}{ }}\label{section-5}

\newpage
\section{Best Practice: Establish an authorship
policy}\label{best-practice-establish-an-authorship-policy}

\textbf{Why we recommend this}: Establishing a policy dedicated to
authorship ensures that people are given due credit for their work. It
also serves as a document that administrators can turn to in case
authorial disputes arise and allows for proactive problem mitigation,
rather than having to resort to reactive interpretation. Further, having
an authorship policy is in keeping with similar policies by journals and
publishers. Having such explicit authorship policies is thus part of a
larger trend. Note that the authorship policy will be at least partially
communicated to users through guidance provided to software
contributors.

\textbf{This should describe}:

\begin{itemize}
\item
  \begin{quote}
  Who should be listed as an author of the software
  \end{quote}
\item
  \begin{quote}
  Policies around making changes to authorship
  \end{quote}
\item
  \begin{quote}
  How authorship disputes are handled
  \end{quote}
\item
  \begin{quote}
  What the resource will do in case of conflict
  \end{quote}
\end{itemize}

\textbf{What you might consider when writing an authorship policy:}

\begin{itemize}
\item
  \begin{quote}
  Who should be listed as an author of the software, taking into
  consideration whether those who are not coders, such as software
  testers or documentation maintainers, will be identified or credited,
  as well as criteria for ordering the list of authors in cases of
  multiple authors
  \end{quote}
\item
  \begin{quote}
  How the resource handles large numbers of authors and group or
  consortium authorship
  \end{quote}
\item
  \begin{quote}
  Including guidelines about how changes to authorship are handled
  \end{quote}
\item
  \begin{quote}
  What role the registry will play, if any, in authorship disputes, and
  if so, how they are handled
  \end{quote}
\item
  \begin{quote}
  Maintaining consistency with the citation policies for the
  registry/repository
  \end{quote}
\item
  \begin{quote}
  Using a credit ontology (\emph{e.g.},
  \href{https://casrai.org/credit/}{\emph{CRediT Contributor Roles
  Ontology}}) to describe authors' contributions
  \end{quote}
\end{itemize}

\section{\texorpdfstring{\\
}{ }}\label{section-6}

\newpage
\section{Best Practice: Share your metadata
schema}\label{best-practice-share-your-metadata-schema}

\textbf{Why we recommend this}: For individual and organizational users
interested in the information in registries and repositories, revealing
the metadata schema used for the entries helps users understand the
structure and properties of the deposited information. The metadata
structure helps to inform users how they might interact with or ingest
records in the collection. A metadata schema mapped to other schemas and
an API specification can improve the interoperability between registries
and repositories.

\textbf{This should describe}:

\begin{itemize}
\item
  \begin{quote}
  What schema is used (e.g.,
  \href{https://codemeta.github.io/}{\emph{CodeMeta}},
  \href{https://schema.org/}{\emph{Schema.org}}) and its version number
  if a published standard schema is used, or, if a custom schema is
  used, a description of the schema and/or a data dictionary
  \end{quote}
\item
  \begin{quote}
  Where the metadata documentation or its official site can be found
  (typically a link)
  \end{quote}
\item
  \begin{quote}
  What metadata is expected when submitting software, including which
  fields are required and which are optional, and the format of the
  content in each field
  \end{quote}
\end{itemize}

\textbf{What you might consider when stating your metadata schema:}

\begin{itemize}
\item
  \begin{quote}
  Using a metadata schema that is mapped (``cross-walked'') to published
  standard schemas, or providing a cross-walk between your schema and
  other schemas
  \end{quote}
\item
  \begin{quote}
  Providing an example of the metadata schema with a complete entry in
  your repository that illustrates all the fields of the schema
  \end{quote}
\end{itemize}

\newpage
\section{Best Practice: Stipulate conditions of use
}\label{best-practice-stipulate-conditions-of-use}

\textbf{Why we recommend this}: A conditions of use policy lets users of
your resource know how the metadata of the registry or repository can be
used, attributed, and/or cited. It provides information about licensing
and forestalls any potential difficulties and/or liabilities, such as
claims of damage for misinterpretation or misapplication of metadata,
that may arise. In turn, it clearly states how the metadata can and
cannot be used, including for commercial purposes and in aggregate form.

\textbf{This should describe}:

\begin{itemize}
\item
  \begin{quote}
  Legal disclaimers about the responsibility and liability borne by the
  registry or repository
  \end{quote}
\item
  \begin{quote}
  License and copyright information, both for individual entries and for
  the registry or repository as a whole
  \end{quote}
\item
  \begin{quote}
  Conditions for the use of the metadata, including prohibitions, if any
  \end{quote}
\item
  \begin{quote}
  Preferred format for citing software entries
  \end{quote}
\item
  \begin{quote}
  Preferred format for attributing or citing the resource itself
  \end{quote}
\end{itemize}

\textbf{What you might consider when writing a conditions of use
policy:}

\begin{itemize}
\item
  \begin{quote}
  What license governs your metadata, and whether there are licensing
  requirements for findings and/or derivatives of the resource
  \end{quote}
\item
  \begin{quote}
  Whether there are differences in the terms and license for commercial
  versus noncommercial use
  \end{quote}
\item
  \begin{quote}
  Conditions for the use of the API if one is available
  \end{quote}
\item
  \begin{quote}
  Restrictions on use of the metadata
  \end{quote}
\item
  \begin{quote}
  Including a statement to the effect that the registry or repository
  makes no guarantees about completeness and is not liable for any
  damages that may arise from the use of the information
  \end{quote}
\end{itemize}

\section{\texorpdfstring{\\
}{ }}\label{section-7}

\newpage
\section{Best Practice: State a privacy
policy}\label{best-practice-state-a-privacy-policy}

\textbf{Why we recommend this:} Having a privacy policy demonstrates a
strong commitment to the privacy of users of the registry or repository,
and allows the resource to comply with the legal requirement of many
countries, in addition to those a home institution and/or funding
agencies may impose. A privacy policy discloses what information,
analytics, and metrics a registry collects and/or retains about its
users and why.

\textbf{This should describe}:

\begin{itemize}
\item
  \begin{quote}
  What information is collected and how long it is retained
  \end{quote}
\item
  \begin{quote}
  How the information, especially any personal data, is used
  \end{quote}
\item
  \begin{quote}
  Whether tracking is done, what is tracked, and how (e.g., Google
  Analytics)
  \end{quote}
\item
  \begin{quote}
  Whether cookies are used
  \end{quote}
\end{itemize}

\textbf{What you might consider when writing a privacy policy:}

\begin{itemize}
\item
  \begin{quote}
  Detailing the specific data collected, why it is collected, and
  whether it is shared or sold
  \end{quote}
\item
  \begin{quote}
  Being explicit about third party tools used to collect analytic
  information and potentially referencing their privacy policies
  \end{quote}
\item
  \begin{quote}
  Stating whether users will receive email as a result of visiting or
  downloading content
  \end{quote}
\item
  \begin{quote}
  Explaining the measures taken to protect users' privacy, and whether
  the resource complies with the
  \href{https://gdpr-info.eu/}{\emph{European Union Directive on General
  Data Protection Regulation}} (GDPR) or other local laws, if applicable
  \end{quote}
\item
  \begin{quote}
  Reserving the right to make changes to the Privacy Policy
  \end{quote}
\item
  \begin{quote}
  Defining a mechanism by which users can request information be removed
  \end{quote}
\end{itemize}

\section{\texorpdfstring{\\
}{ }}\label{section-8}

\newpage
\section{Best Practice: Provide a retention
policy}\label{best-practice-provide-a-retention-policy}

\textbf{Why we recommend this}: Software registries and repositories
make an implicit promise to retain records for some period of time, but
for various reasons may have to remove records. Common examples include
removing entries that are outdated or no longer meet the scope of the
registry or are found to be in violation of policies. The collection
should document retention goals so that users and depositors are aware
of them.

\textbf{This should describe}:

\begin{itemize}
\item
  \begin{quote}
  The length of time metadata and/or files are expected to be retained
  \end{quote}
\item
  \begin{quote}
  Under what conditions metadata and/or files are removed
  \end{quote}
\item
  \begin{quote}
  Who has the responsibility and ability to remove information
  \end{quote}
\item
  \begin{quote}
  Procedures to request that metadata and/or files be removed
  \end{quote}
\end{itemize}

\textbf{What you might consider when writing a retention policy:}

\begin{itemize}
\item
  \begin{quote}
  If assigning identifiers, whether best practices for persistent
  identifiers are followed, including resolvability, retention, and
  non-reuse of those identifiers
  \end{quote}
\item
  \begin{quote}
  Making sure the length of time is not too prescriptive (e.g., ``for
  the next 10 years''), but rather fits within the context of the
  underlying organization(s) and its funding
  \end{quote}
\item
  \begin{quote}
  Stating who is allowed to edit metadata, delete records, or delete
  files, and if so, how these changes are documented and consistent with
  the registry broadly
  \end{quote}
\item
  \begin{quote}
  Explaining the process by which data may be taken offline and archived
  as well as the process for its possible retrieval
  \end{quote}
\end{itemize}

\section{\texorpdfstring{\\
}{ }}\label{section-9}

\newpage
\section{Best Practice: Disclose your end-of-life
policy}\label{best-practice-disclose-your-end-of-life-policy}

\textbf{Why we recommend this}: Sharing a clear end-of-life policy
increases trust in the community served by your registry or repository.
It demonstrates a thoughtful commitment to users by informing them that
provisions for the artifacts contained in the resource have been
considered should the resource close or otherwise end its services for
these artifacts. Such a policy sets expectations and provides
reassurance as to how long the records within the resource will be
findable and accessible in the future.

\textbf{This should describe}:

\begin{itemize}
\item
  \begin{quote}
  Under what circumstances the resource might end its services
  \end{quote}
\item
  \begin{quote}
  What consequences would result from closure
  \end{quote}
\item
  \begin{quote}
  What will happen to the metadata and/or the software artifacts
  contained in the resource in the event of closure
  \end{quote}
\item
  \begin{quote}
  If long-term preservation is expected, where metadata and/or software
  artifacts will be migrated for preservation
  \end{quote}
\item
  \begin{quote}
  How a migration will be funded
  \end{quote}
\end{itemize}

\textbf{What you might consider when writing a end-of-life policy:}

\begin{itemize}
\item
  \begin{quote}
  Whether the records will remain available, and if so, how and for
  whom, and under which conditions, such as archived status or ``read
  only'', should the collection close
  \end{quote}
\item
  \begin{quote}
  What restrictions, if any, may apply
  \end{quote}
\item
  \begin{quote}
  Establishing a formal agreement or MOU with another registry,
  repository, or institution to receive and preserve the data or
  project, if applicable
  \end{quote}
\end{itemize}

\section{\texorpdfstring{\\
}{ }}\label{section-10}

\newpage
\section{Policy examples}\label{policy-examples}

\subsection{Scope Statement}\label{scope-statement}

\begin{quote}
Astrophysics Source Code Library. (n.d.). \emph{Editorial policy}.
\href{https://ascl.net/wordpress/submissions/editiorial-policy/}{\emph{https://ascl.net/wordpress/submissions/editiorial-policy/}}

bio.tools. (n.d.). \emph{Curators Guide}.
\href{https://biotools.readthedocs.io/en/latest/curators_guide.html}{\emph{https://biotools.readthedocs.io/en/latest/curators\_guide.html}}

Caltech Library. (2017). \emph{Terms of Deposit}.
\href{https://data.caltech.edu/terms}{\emph{https://data.caltech.edu/terms}}

Caltech Library. (2019). \emph{CaltechDATA FAQ}. Caltech Library.
\href{https://www.library.caltech.edu/caltechdata/faq}{\emph{https://www.library.caltech.edu/caltechdata/faq}}

Computational Infrastructure for Geodynamics. (n.d.). \emph{Code
Donation}.
\href{https://geodynamics.org/cig/dev/code-donation/}{\emph{https://geodynamics.org/cig/dev/code-donation/}}

CoMSES Net Computational Model Library. (n.d.). \emph{Frequently Asked
Questions}.
\href{https://www.comses.net/about/faq/\#model-library}{\emph{https://www.comses.net/about/faq/\#model-library}}

ORNL DAAC for Biogeochemical Dynamics. (n.d.). \emph{Data Scope and
Acceptance Policy}.
\href{https://daac.ornl.gov/submit/}{\emph{https://daac.ornl.gov/submit/}}

RDA Registry and Research Data Australia. (2018). \emph{Collection}.
ARDC Intranet.
\href{https://intranet.ands.org.au/display/DOC/Collection}{\emph{https://intranet.ands.org.au/display/DOC/Collection}}

Remote Sensing Code Library. (n.d.). \emph{Submit}.
\href{https://rscl-grss.org/submit.php}{\emph{https://rscl-grss.org/submit.php}}

SciCrunch. (n.d.). \emph{Curation Guide for SciCrunch Registry}.
\href{https://scicrunch.org/page/Curation\%20Guidelines}{\emph{https://scicrunch.org/page/Curation\%20Guidelines}}

U.S. Department of Energy: Office of Scientific and Technical
Information. (n.d.-a). \emph{DOE CODE: Software Policy}.
\href{https://www.osti.gov/doecode/policy}{\emph{https://www.osti.gov/doecode/policy}}

U.S. Department of Energy: Office of Scientific and Technical
Information. (n.d.-b). \emph{FAQs}. OSTI.GOV.
\href{https://www.osti.gov/faqs}{\emph{https://www.osti.gov/faqs}}
\end{quote}

\subsection{Authorship}\label{authorship}

\begin{quote}
CASRAI. (n.d.). CRediT - Contributor Roles Taxonomy. \emph{CASRAI}.
\href{https://casrai.org/credit/}{\emph{https://casrai.org/credit/}}

Committee on Publication Ethics: COPE. (2020a). \emph{Authorship and
contributorship}.
\href{https://publicationethics.org/authorship}{\emph{https://publicationethics.org/authorship}}

Committee on Publication Ethics: COPE. (2020b). \emph{Core practices}.
\href{https://publicationethics.org/core-practices}{\emph{https://publicationethics.org/core-practices}}

Dagstuhl EAS Specification Draft. (2016). \emph{The Software Credit
Ontology}.
\href{https://dagstuhleas.github.io/SoftwareCreditRoles/doc/index-en.html\#}{\emph{https://dagstuhleas.github.io/SoftwareCreditRoles/doc/index-en.html\#}}

Journal of Open Source Software. (n.d.). \emph{Ethics Guidelines}.
\href{https://joss.theoj.org/about\#ethics}{\emph{https://joss.theoj.org/about\#ethics}}

ORNL DAAC (n.d) \emph{Authorship Policy}.
\emph{https://daac.ornl.gov/submit/}

PeerJ Journals. (n.d.-a). \emph{Author Policies}.
\href{https://peerj.com/about/policies-and-procedures/\#author-policies}{\emph{https://peerj.com/about/policies-and-procedures/\#author-policies}}

PeerJ Journals. (n.d.-b). \emph{Publication Ethics}.
\href{https://peerj.com/about/policies-and-procedures/\#publication-ethics}{\emph{https://peerj.com/about/policies-and-procedures/\#publication-ethics}}

PLOS ONE. (n.d.). \emph{Authorship}.
\href{https://journals.plos.org/plosone/s/authorship}{\emph{https://journals.plos.org/plosone/s/authorship}}

The Contributor Role Ontology,
\href{https://github.com/data2health/contributor-role-ontology}{\emph{https://github.com/data2health/contributor-role-ontology}}
\end{quote}

\subsection{Metadata Schema}\label{metadata-schema}

\begin{quote}
ANDS: Australian National Data Service. (n.d.). \emph{Metadata}. ANDS.
\href{https://www.ands.org.au/working-with-data/metadata}{\emph{https://www.ands.org.au/working-with-data/metadata}}

ANDS: Australian National Data Service. (2016). \emph{ANDS Guide:
Metadata}.
\href{https://www.ands.org.au/__data/assets/pdf_file/0004/728041/Metadata-Workinglevel.pdf}{\emph{https://www.ands.org.au/\_\_data/assets/pdf\_file/0004/728041/Metadata-Workinglevel.pdf}}

Bernal, I. (2019). \emph{Metadata for Data Repositories}.
\href{https://doi.org/10.5281/zenodo.3233486}{\emph{https://doi.org/10.5281/zenodo.3233486}}

bio.tools. (2020). \emph{Bio-tools/biotoolsSchema} {[}HTML{]}.
bio.tools.
\href{https://github.com/bio-tools/biotoolsSchema}{\emph{https://github.com/bio-tools/biotoolsSchema}}
(Original work published 2015)

bio.tools. (2019). \emph{BiotoolsSchema documentation}.
\href{https://biotoolsschema.readthedocs.io/en/latest/}{\emph{https://biotoolsschema.readthedocs.io/en/latest/}}

The CodeMeta crosswalks. (n.d.)
\href{https://codemeta.github.io/crosswalk/}{\emph{https://codemeta.github.io/crosswalk/}}

Citation File Format (CFF). (n.d.)
\href{https://doi.org/10.5281/zenodo.1003149}{\emph{https://doi.org/10.5281/zenodo.1003149}}

The DataVerse Project. (2020). DataVerse 4.0+ Metadata Crosswalk.
\href{https://docs.google.com/spreadsheets/d/10Luzti7svVTVKTA-px27oq3RxCUM-QbiTkm8iMd5C54/edit\#gid=587531992}{\emph{https://docs.google.com/spreadsheets/d/10Luzti7svVTVKTA-px27oq3RxCUM-QbiTkm8iMd5C54/edit\#gid=587531992}}

OntoSoft. (2015). \emph{OntoSoft Ontology}.
\href{https://ontosoft.org/ontology/software/}{\emph{https://ontosoft.org/ontology/software/}}

OpenAPI Specification. (2020).
\href{http://spec.openapis.org/oas/v3.0.3}{\emph{http://spec.openapis.org/oas/v3.0.3}}

Zenodo. (n.d.-a). \emph{Schema for Depositing}.
\href{https://zenodo.org/schemas/records/record-v1.0.0.json}{\emph{https://zenodo.org/schemas/records/record-v1.0.0.json}}

Zenodo. (n.d.-b). \emph{Schema for Published Record}.
\href{https://zenodo.org/schemas/deposits/records/legacyrecord.json}{\emph{https://zenodo.org/schemas/deposits/records/legacyrecord.json}}
\end{quote}

\subsection{Conditions of use policy}\label{conditions-of-use-policy}

\begin{quote}
Allen Institute. (n.d.). \emph{Terms of Use}.
\href{https://alleninstitute.org/legal/terms-use/}{\emph{https://alleninstitute.org/legal/terms-use/}}

Europeana. (n.d.). \emph{Usage Guidelines for Metadata}. Europeana
Collections.
\href{https://www.europeana.eu/portal/en/rights/metadata.html}{\emph{https://www.europeana.eu/portal/en/rights/metadata.html}}

U.S. Department of Energy: Office of Scientific and Technical
Information. (n.d.). \emph{DOE CODE FAQ: Are there restrictions on the
use of the material in DOE CODE?}
\href{https://www.osti.gov/doecode/faq\#are-there-restrictions}{\emph{https://www.osti.gov/doecode/faq\#are-there-restrictions}}

Zenodo. (n.d.). \emph{Terms of Use}.
\href{https://about.zenodo.org/terms/}{\emph{https://about.zenodo.org/terms/}}
\end{quote}

\subsection{Privacy policy}\label{privacy-policy}

\begin{quote}
Allen Institute. (n.d.). \emph{Privacy Policy}.
\href{https://alleninstitute.org/legal/privacy-policy/}{\emph{https://alleninstitute.org/legal/privacy-policy/}}

CoMSES Net. (n.d.). \emph{Data Privacy Policy.
\url{https://www.comses.net/about/data-privacy/}}

Nature. (2020). \emph{Privacy Policy}.
\href{https://www.nature.com/info/privacy}{\emph{https://www.nature.com/info/privacy}}

Research Data Australia. (n.d.). \emph{Privacy Policy}. Research Data
Australia.
\href{https://researchdata.ands.org.au/page/privacy}{\emph{https://researchdata.ands.org.au/page/privacy}}

SciCrunch. (2018). \emph{Privacy Policy}. SciCrunch.
\href{https://scicrunch.org/page/privacy}{\emph{https://scicrunch.org/page/privacy}}

Science Repository. (n.d.). \emph{Privacy Policies}.
\href{https://www.sciencerepository.org/privacy}{\emph{https://www.sciencerepository.org/privacy}}

Zenodo. (n.d.). \emph{Privacy policy}.
\href{https://about.zenodo.org/privacy-policy/}{\emph{https://about.zenodo.org/privacy-policy/}}
\end{quote}

\subsection{Retention Policy}\label{retention-policy}

\begin{quote}
Caltech Library. (n.d.). \emph{CaltechDATA FAQ}. Caltech Library.
\href{https://www.library.caltech.edu/caltechdata/faq}{\emph{https://www.library.caltech.edu/caltechdata/faq}}

CoMSES Net Computational Model Library. (n.d.). \emph{How long will
models be stored in the Computational Model Library?
\href{https://www.comses.net/about/faq/}{\emph{https://www.comses.net/about/faq/}}}

Dryad. (2020). \emph{Dryad FAQ - Publish and Preserve your Data}.
\href{https://datadryad.org/stash/faq\#preserved}{\emph{https://datadryad.org/stash/faq\#preserved}}

Software Heritage. (n.d.). \emph{Content policy}.
\href{https://www.softwareheritage.org/legal/content-policy/}{\emph{https://www.softwareheritage.org/legal/content-policy/}}

Zenodo. (n.d.). \emph{General Policies v1.0}.
\href{https://about.zenodo.org/policies/}{\emph{https://about.zenodo.org/policies/}}

Bioconductor. (2020). \emph{Package End of Life Policy.
\url{https://bioconductor.org/developers/package-end-of-life/} }
\end{quote}

\subsection{End-of-life policy}\label{end-of-life-policy}

\begin{quote}
Figshare. (n.d.). \emph{Preservation and Continuity of Access Policy}.
\href{https://knowledge.figshare.com/articles/item/preservation-and-continuity-of-access-policy}{\emph{https://knowledge.figshare.com/articles/item/preservation-and-continuity-of-access-policy}}

Open Science Framework. (2019). \emph{FAQs}. OSF Guides.
\href{http://help.osf.io/hc/en-us/articles/360019737894-FAQs}{\emph{http://help.osf.io/hc/en-us/articles/360019737894-FAQs}}

NASA Earth Science Data Preservation Content Specification (n.d.)
\emph{\url{https://earthdata.nasa.gov/esdis/eso/standards-and-references/preservation-content-spec}
}

Zenodo. (n.d.). \emph{Frequently Asked Questions}.
\href{https://help.zenodo.org/}{\emph{https://help.zenodo.org/}}
\end{quote}

\section{\texorpdfstring{\\
}{ }}\label{section-11}

\newpage
\section{Additional useful sites }\label{additional-useful-sites}

In addition to the links to sites and information embedded in this
Concise Guide, the following sites are directly applicable to the best
practices we have listed.

{[}Authorship{]} Citation File Format:
\href{https://citation-file-format.github.io/}{\emph{https://citation-file-format.github.io/}}

{[}Authorship{]} CiteAs:
\href{https://citeas.org/}{\emph{https://citeas.org/}}

{[}Metadata Schema{]} Software Heritage Metadata workflow:
\href{https://docs.softwareheritage.org/devel/swh-indexer/metadata-workflow.html}{\emph{https://docs.softwareheritage.org/devel/swh-indexer/metadata-workflow.html}}

{[}Metadata Schema{]} W3C data profile definition:
\href{https://www.w3.org/TR/dx-prof-conneg/\#dfn-data-profile}{\emph{https://www.w3.org/TR/dx-prof-conneg/\#dfn-data-profile}}

You may also be interested in the
\href{https://www.coretrustseal.org/why-certification/requirements/}{\emph{Core
Trustworthy Data Repositories requirements}}, which are intended to
reflect the characteristics of trustworthy repositories.

\newpage
\section{Glossary}\label{glossary}

\textbf{API}: \href{https://en.wikipedia.org/wiki/API}{\emph{Application
programming interface}}

\textbf{Collection}: Used in this document as a synonym for
\emph{registries and repositories}

\textbf{Depositor}: A user who submits information and/or software to a
registry or repository; synonymous with \emph{software contributor}

\textbf{Entry}: Information about and/or software for a particular
holding in a registry or repository; synonymous with \emph{record}

\textbf{JSON-LD}:
\href{https://en.wikipedia.org/wiki/JSON-LD}{\emph{JavaScript Object
Notation for Linked Data}}

\textbf{Metadata}: \emph{Information about a code or software package}

\textbf{Record}: Information about and/or software for a particular
holding in a registry or repository; synonymous with \emph{entry}

\textbf{Registry}: Typically an index or catalog of software stored
elsewhere

\textbf{Repository}: Typically a site that both indexes and stores
software

\textbf{Resource}: Used in this document as a synonym for
\emph{registries and repositories}

\textbf{Software author}: A person who is credited as an author of a
software package; this may include not only one who writes code, but
also one who tests, documents, maintains, or otherwise contributes
effort to the software package

\textbf{Software contributor}: A user who submits information and/or
software to a registry or repository; synonymous with \emph{depositor}

\textbf{XML}: \href{https://en.wikipedia.org/wiki/XML}{\emph{Extensible
Markup Language}}

\newpage
\section{Authors}\label{authors}

Alain Monteil, INRIA: HAL/Software Heritage,
\href{https://orcid.org/0000-0003-3150-4837}{\emph{https://orcid.org/0000-0003-3150-4837}}

Alejandra Gonzalez-Beltran, Science and Technology Facilities Council,
UK Research and Innovation:
\href{https://orcid.org/0000-0003-3499-8262}{\emph{https://orcid.org/0000-0003-3499-8262}}

Alexandros Ioannidis, CERN: Zenodo,
\href{https://orcid.org/0000-0002-5082-6404}{\emph{https://orcid.org/0000-0002-5082-6404}}

Alice Allen, Astrophysics Source Code Library,
\href{https://orcid.org/0000-0003-3477-2845}{\emph{https://orcid.org/0000-0003-3477-2845}}

Allen Lee, Arizona State University: CoMSES Net,
\href{https://orcid.org/0000-0002-6523-6079}{\emph{https://orcid.org/0000-0002-6523-6079}}

Anita Bandrowski, UCSD: SciCrunch,
\href{https://orcid.org/0000-0002-5497-0243}{\emph{https://orcid.org/0000-0002-5497-0243}}

Bruce E. Wilson, Oak Ridge National Laboratory, ORNL Distributed Active
Archive Center for Biogeochemical Dynamics,
\href{https://orcid.org/0000-0002-1421-1728}{\emph{https://orcid.org/0000-0002-1421-1728}}

Bryce Mecum, NCEAS, UC Santa Barbara: CodeMeta,
\href{https://orcid.org/0000-0002-0381-3766}{\emph{https://orcid.org/0000-0002-0381-3766}}

Cai Fan Du, iSchool, University of Texas at Austin: CiteAs,
\href{https://orcid.org/0000-0003-2538-607X}{\emph{https://orcid.org/0000-0003-2538-607X}}

Carly Robinson, DOE-OSTI,
\href{https://orcid.org/0000-0002-8523-1478}{\emph{https://orcid.org/0000-0002-8523-1478}}

Daniel S. Katz, University of Illinois at Urbana-Champaign, Associate
EiC for JOSS: FORCE11 Software Citation Implementation Working Group
co-chair,
\href{https://orcid.org/0000-0001-5934-7525}{\emph{https://orcid.org/0000-0001-5934-7525}}

Daniel Garijo, Information Sciences Institute, University of Southern
California: Ontosoft,
\href{http://orcid.org/0000-0003-0454-7145}{\emph{http://orcid.org/0000-0003-0454-7145}}

David Long, Brigham Young University: IEEE GRS Remote Sensing Code
Library:
\href{https://orcid.org/0000-0002-1852-3972}{\emph{https://orcid.org/0000-0002-1852-3972}}

Genevieve Milliken, Health Sciences Library, New York University,
\href{https://orcid.org/0000-0002-3057-0659}{\emph{https://orcid.org/0000-0002-3057-0659}}

Hervé Ménager, Institut Pasteur: ELIXIR bio.tools,
\href{https://orcid.org/0000-0002-7552-1009}{\emph{https://orcid.org/0000-0002-7552-1009}}

Jessica Hausman, JPL PO.DAAC,
\href{https://orcid.org/0000-0002-1861-1526}{\emph{https://orcid.org/0000-0002-1861-1526}}

Jurriaan H. Spaaks, Netherlands eScience Center: Research Software
Directory,
\href{https://orcid.org/0000-0002-7064-4069}{\emph{https://orcid.org/0000-0002-7064-4069}}

Katrina Fenlon, University of Maryland: iSchool,
\href{https://orcid.org/0000-0003-1483-5335}{\emph{https://orcid.org/0000-0003-1483-5335}}

Kristin Vanderbilt, Environmental Data Initiative: IMCR,
\href{https://orcid.org/0000-0003-1439-2204}{\emph{https://orcid.org/0000-0003-1439-2204}}

Lorraine Hwang, Computational Infrastructure for Geodynamics, UC Davis,
\href{http://orcid.org/0000-0002-1021-3101}{\emph{http://orcid.org/0000-0002-1021-3101}}

Lynn Davis, DOE-OSTI, \href{https://orcid.org/0000-0002-4670-0964}{\emph{https://orcid.org/0000-0002-4670-0964}}

Martin Fenner, DataCite, FORCE11 Software Citation Implementation
Working Group co-chair,
\href{https://orcid.org/0000-0003-1419-2405}{\emph{https://orcid.org/0000-0003-1419-2405}}

Michael R. Crusoe, CWL; Debian-Med,
\href{https://orcid.org/0000-0002-2961-9670}{\emph{https://orcid.org/0000-0002-2961-9670}}

Mike Hucka, Caltech, Caltech, SBML, COMBINE,
\href{https://orcid.org/0000-0001-9105-5960}{\emph{https://orcid.org/0000-0001-9105-5960}}

Mingfang Wu, Australian Research Data Commons,
\href{https://orcid.org/0000-0003-1206-3431}{\emph{https://orcid.org/0000-0003-1206-3431}}

Neil Chue Hong, SSI, FORCE11 Software Citation Implementation Working
Group co-chair,
\href{https://orcid.org/0000-0002-8876-7606}{\emph{https://orcid.org/0000-0002-8876-7606}}

Peter Teuben, UMD,
\href{https://orcid.org/0000-0003-1774-3436}{\emph{https://orcid.org/0000-0003-1774-3436}}

Shelley Stall, AGU,
\href{https://orcid.org/0000-0003-2926-8353}{\emph{https://orcid.org/0000-0003-2926-8353}}

Stephan Druskat, German Aerospace Center (DLR)/University
Jena/Humboldt-Universität zu Berlin: Citation File Format,
\href{https://orcid.org/0000-0003-4925-7248}{\emph{https://orcid.org/0000-0003-4925-7248}}

Ted Carnevale, Neuroscience Department, Yale: ModelDB

Tom Morrell, Caltech: CaltechDATA,
\href{https://orcid.org/0000-0001-9266-5146}{\emph{https://orcid.org/0000-0001-9266-5146}}

\end{document}
